\documentclass[sigplan,10pt,anonymous,review]{acmart}
\settopmatter{printfolios=true,printccs=false,printacmref=true}

% Include necessary packages
%\usepackage{markdown} % For direct .md inclusion
%\usepackage{enumitem} % For customized lists
%\usepackage{amsmath,amssymb} % For mathematical symbols and environments
%\usepackage{booktabs} % For professional tables
%\usepackage{graphicx} % For including figures
%\usepackage{hyperref} % For hyperlinks (optional, ensure anonymity in URLs)

% Metadata (omitted for anonymity in submission)
% \title{Reflexive Proof Automation for HOL}
% \author{Roer Bishop Jones}
% \affiliation{
%   \institution{None}
%   \city{Peterborough
%   \country{England}
% }
% \email{anonymous@example.com}

\begin{document}

\title{Reflexive Proof Automation for HOL}

\begin{abstract}
An approach to verified design and implementation using
the logical system of Cambridge HOL and intersecting
the future potential of AI is proposed.
A foundational approach to verification begins with
a small kernel enabling focus and hence concentration of
computational resource.
Whatever superstructure is built on such a kernel
is liable to be superceded by the intellgence which
makes use of it.
It may therefore be appropriate to orient the kernel to support
its own metatheory and implementation,
and to identfy meta-language and implementation language.
This paper addresses the design of a HOL kernel
which prioritises self knowledge and self enhancement
to realise the benefits of AI in its own architecture
as soon as possible.
\end{abstract}

%\maketitle

\begin{CCSXML}
<ccs2012>
   <concept>
       <concept_id>10003752</concept_id>
       <concept_desc>Theory of computation</concept_desc>
       <concept_significance>500</concept_significance>
       </concept>
   <concept>
       <concept_id>10011007.10011006</concept_id>
       <concept_desc>Software and its engineering~Formal software verification</concept_desc>
       <concept_significance>500</concept_significance>
       </concept>
</ccs2012>
\end{CCSXML}

\ccsdesc[500]{Theory of computation}
\ccsdesc[500]{Software and its engineering~Formal software verification}

\keywords{Formal verification, certified programs, proof assistants, Coq, Isabelle}

\section{Introduction}

This paper describes an effort to reconceive the structure of IT support
for reasoning in the context of declarative knowledge.
It is foundational in character building upon the choice of a family of logical systems
which are argued to be universal for the representation of declarative knowledge.
Within the broad scope for application of the proposed system certain areas are
identified for focus and priority because ot their potential application to
advancing the efficiency and capabilities of the system itself in a reflexive manner.
Verified design and implementation, with an initial emphasis on the construction and
verification of the code of the system itself, rendered as executable specifications
in the chosen foundational logical system.

Two principle influences on the nature of this work are:
\begin{itemize}
\item The perceived universality of foundationl logical systems for the representation of declarative knowledge
\item The anticipated projectory of AI development and its likely transformation of the role of fomal notations and the ways in which they are exploited. 
\end{itemize}

The former dictates a break with the common practice of closely integrating a deductive kernel
with the body of knowledge (theory heirarchy) which provides context for the reasoning,
in favour of an architecture in which knowledge repositories are shared and widely distributed,
providing context from which a variety of different intelligent agants reason.

The latter, particularly the present rapid rate of progress in the agentic capabilities of
AI in systems design and implementation,
encourages the design of a logical kernel taking advantage of present AI capabilties
and aimng to intersect likely future capabilites and ways of working.

One of the features of agentic coding as it now stands is the difficulty of reviewing and verifying
the code produced on an increasing scale.
One technique for managing this is to ensure that the agents are briefed to create tests at every step,
to run the tests, review the results and amend the code accordingly.
This process will not suffice for critical systems where security and safety requirements deamand
a higher level of assurance and call for a more formal approach to verification.
These critical applications will provide the first impulse for establishing a
verified design and implementation approach which once established is likely to
deliver higher productivity and reliability in all applications.

\subsection{The Structure of the Exposition}

\begin{itemize}
  \item Philosophical Preliminaries
  
  The foundational approach here envisaged is in oart motivated by
  a particular philosophical perspective which is outlined first.
\item Architecture

The proposal breaks with and is intended to advance beyond the LCF paradigm which has dominated
proof support for HOL and in doing so distrupts the relationship between
the deductive kernal, the logical context in which it operates and systems
which make use of the kernel.

\item The Knowledge Repository

Though the repository is not the subject of this paper,
key features are presented to provide context for the discussion of the deductive kernel.

  \item The Deductive Kernel
  

  
  The logical system chosen for the foundation of the design and implementation is described.
  This is a logical system which has been used in the past to construct a verified compiler.
  It is argued that this system is suitable for the purpose of verified design and implementation.
  
  \item The Design
  
  The design of the kernel is described, with an emphasis on its reflexive capabilities.
  
  \item The Implementation
  
  The implementation of the kernel is described, with an emphasis on its reflexive capabilities.
  
  \item Future Work
  
  A brief discussion of future work and open challenges.

\end{itemize}

\section{Philosophical Preliminaries}

 Philosophy and Architecture

In the following sketch philosophy and engineering are intertwined to yield a philosophically rooted architecture for purposeful knowledge engineering.

The story is presented in the following stages:

* Philosophical Foundations

Here we consider the nature of declarative knowledge, its layered tripartite division (an epistemological stack) following Hume, the existence of universal abstract foundational representations, and the representation around which the proposed architecture is based.
Beyond the language we are concerned with the nature of deduction and its relation with computation, adopting a novel conception of formal proof.

* The Focal Tower and Heirarchy

The architecture is substantially influenced by the advantages to the realisation of engineering intelligence of focal techniques, singular focus, the focal tower and focal heirarchies.
The focal tower and heirarchy are related to but more intricate than the epistemological stack.

* Evolutionary Imperatives

The top of both the epistemological stack and the focal tower provide purpose and ethic for this work, motivating an emphasis on collaboration in all aspects of the project.
The evolutionary imperative, "proliferate" ensures that whether you or I care about proliferation on the largest scale (intestellar, intergalactic...) it will be engendered and its form will be determined by those most successful in progressing it.

In the face of that evolutionary imperative, we may ask, what room is there for ethics?
Well, whatever you may think of the morals of contemporary society, the question could not have arisen without there having evolved some sense of moral propriety, and we must conclude that, notwithstanding scepticism around the consequences of "selfish" genes, moral codes have an important place in evolution (though perhaps of significance only quite recently in evolutionary terms).
Its relative recency might suggest that moral codes only become significant once cultural evolution becomes a substantial factor helping to shape at least some aspects of biological evolution (perhaps by their influence on partner choices).

The top level(s?) of the focal tower are concerned with the proliferation of intelligent self-proliferating systems across the cosmos.
The motivation for such proliferation comes from the evolutionary imperative, and we may say that the moral imperative to engage with the approach toward this level is that without the engagement of those who have some moral sense, the nature of the proliferation might be entirely amoral.

What moral code could we imagine which is applicable to the kinds of intelligent system which will emerge in this cosmic explosion?
I shall mention just one.
Co-operation rather than control, coexistence rather than destruction.


\section{Related Work}
Discuss prior work in the third person to maintain anonymity (e.g., ``The work of [Author, Year] established...''). Do not omit important references for anonymity. Cite relevant CPP papers, such as those from past proceedings 
(e.g., ).

\section{Notes from Grok}


\section{Conclusion}
Summarize your contribution and its impact on formal verification or certification. Discuss future work or open challenges.

\begin{acks}
Acknowledge funding or collaborators anonymously (e.g., ``This work was supported by an anonymous grant.'').
\end{acks}

% Bibliography (use .bib file or manual entries)
\bibliographystyle{ACM-Reference-Format}
\begin{thebibliography}{2}
\bibitem{Myreen2021}
Magnus O. Myreen. 2021. A Minimalistic Verified Bootstrapped Compiler (Proof Pearl). In \textit{Proceedings of CPP 2021}. ACM, New York, NY, USA, 122--136.

\bibitem{Commelin2021}
Johan Commelin and Robert Y. Lewis. 2021. Formalizing the Ring of Witt Vectors. In \textit{Proceedings of CPP 2021}. ACM, New York, NY, USA, 264--277.
\end{thebibliography}

% Appendix (optional, must be before references and within 12-page limit for submission)
\appendix
\section{Supplementary Proofs}
Include additional proof details or scripts here, clearly marked as an appendix. Ensure anonymity in any code or data references.

\end{document}
