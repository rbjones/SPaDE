% $Id: t039a.tex,v 1.3 2010/08/08 15:50:44 rbj Exp $

\documentclass[11pt]{article}
\usepackage{latexsym}
\usepackage{rbj}

\ftlinepenalty=9999
\usepackage{A4}

% the following two modal operators come from the amsfonts package
\def\PrKI{\Diamond}	%Modify printing for \250
\def\PrKJ{\Box}		%Modify printing for \251

\def\PrJA{\|-}		%Modify printing for  (syntactic entailment)
\def\PrJI{\models}	%Modify printing for ˜ (semantic entailment)

\tabstop=0.4in
\newcommand{\ignore}[1]{}

\def\thyref#1{Appendix \ref{#1}}

%\def\ExpName{\mbox{{\sf exp}}}
%\def\Exp#1{\ExpName(#1)}


\title{Equivalences, Quotients, Universal Algebra and Lattice Theory}
\makeindex
\usepackage[unicode]{hyperref}
\hypersetup{pdfauthor={Roger Bishop Jones}, pdffitwindow=false}
\hypersetup{colorlinks=true, urlcolor=red, citecolor=blue, filecolor=blue, linkcolor=blue}
\author{Roger Bishop Jones}
\date{\ }

\begin{document}
\begin{titlepage}
\maketitle
\begin{abstract}
This is a limited development of universal algebra and lattice theory for the purposes of X-Logic.
\end{abstract}
\vfill

\begin{centering}
{\footnotesize

Created 2010/07/20

Last Change $ $Date: 2010/08/08 15:50:44 $ $

\href{http://www.rbjones.com/rbjpub/pp/doc/t039.pdf}
{http://www.rbjones.com/rbjpub/pp/doc/t039.pdf}

$ $Id: t039a.tex,v 1.3 2010/08/08 15:50:44 rbj Exp $ $

\copyright\ Roger Bishop Jones; Licenced under Gnu LGPL

}%footnotesize
\end{centering}

\thispagestyle{empty}
\end{titlepage}

\newpage
\addtocounter{page}{1}
{\parskip=0pt\tableofcontents}

\section{Prelude}

This document is intended possibly to form a chapter of {\it Analyses of Analysis} \cite{rbjb001}, depending on how comprehensive I decide to make that work.

Some elementary Lattice theory is needed for the lattice of trust or assurance at the heart of ``X-Logic'',
The Universal Algebra is an attempt to do once some basic theory which should be applicable both in Lattice theory and in other algebraic theories.
It is a matter of interest to discover whether this level of abstractions pays off in the development of algebraic theories (in terms of reduced overall complexity and cost).

Further discussion of what might become of this document in the future may be found the postscript (Section \ref{POSTSCRIPT}).

In this document, phrases in coloured text are hyperlinks, like on a web page, which will usually get you to another part of this document (the blue parts, the contents list, page numbers in the Index) but sometimes take you (the red bits) somewhere altogether different (if you happen to be online) like \href{http://rbjones.com/rbjpub/pp/doc/t039.pdf}{online copy of this document}.

\section{Introduction}
